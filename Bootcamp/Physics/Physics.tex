\documentclass[10pt,a4paper]{article}
\usepackage[utf8]{inputenc}
\usepackage[english]{babel}
\usepackage{amsmath}
\usepackage{amsfonts}
\usepackage{amssymb}
\usepackage{graphicx}
\usepackage[left=2cm,right=2cm,top=2cm,bottom=2cm]{geometry}



\author{Baptiste Rouger}
\title{Bootcamp : Physics}



\begin{document}


\maketitle

\begin{flushright}
carles.blanch-mercader@curie.fr
\end{flushright}

\tableofcontents

\newpage

\part{Brownian motion}

Diffusion give rise to \emph{material transport}. It gives a quantitative understanding of many phenomenon (permeability...).

\section{Historical aspects :}
\begin{description}
    \item [1834 :] Clapeyron proposed a phenomenological equation for the ideal gas : $PV = n R T$, \\$PV = N.K_B .T$ (N : number of particules, $K_B$ Boltzman constant $1.38\times 10^{-23} \frac{kg . m}{K . s^2})$
    \item [1828 :] Brown : studied pollen that was moving
    \item [1905 :] Einstein : $\langle x^2\rangle = 2DT$ (D : diffusion coefficient)
    \item [1908 :] Perrin : confirmed the Einstein guess
\end{description}

\section{Random-walk}

Simplest model that describes Brownian motion. A particle has the same probabiblity to move to the right and to the left. After two jumps, there is one way to 2 ($\frac{1}{4}$) and -2 ($\frac{1}{4}$), and 2 ways to 0 ($\frac{1}{2}$).\\
\[ \langle a \rangle = \frac{1}{m} \sum ^m _{i=1} a_i =\sum ^m _{i=1} p_ia_i \]
\[ \langle a^2 \rangle = \frac{1}{m} \sum ^m _{i=1} a_i^2 \]
$\langle a \rangle$ the average position, with m the number of realisations, i the i-th realisation and $a_i$ the i-th realisation.\\
In our example : $\langle a \rangle = \frac{1}{4} (2 + {-2} + 0 + 0) = 0$ and $\langle a^2 \rangle = \frac{1}{4} (2^2 + {-2}^2 + 0^2 + 0^2) = 2$\\

In the 3 jumps case :
\begin{itemize}
    \item 1 way to 3
    \item 1 way to -3
    \item 3 ways to 1
    \item 3 ways to -1
\end{itemize}
$\langle a \rangle = \frac{1}{8}( 3 + {-3} + 3 \times 1 + 3\times {-1} ) = 0$ and $\langle a^2 \rangle = \frac{1}{8}( 3^2 + {-3}^2 + (3 \times 1)^2 + (3\times {-1})^2 )=3$\\

\[ N-jumps = \langle a \rangle = 0\text{ and }\langle a^2 \rangle = N\]

If N is large enough, distribution of position is a binomial distribution.\\

If we say that the boxes have a $l$ length, then $x = a \times l$, $t_0$ is the time delay between successive jumps. So $\frac{t}
{t_0} = N$
\[ \langle x \rangle = \langle a \rangle l = 0 \]
\begin{align*}
    \langle x^2 \rangle &= \langle a^2 \rangle l^2 \\
    &= l^2 N \\
    &= t\frac{l^2}{t_0}\\
    &= 2t\frac{l^2}{2t_0}\\
    So ~ D &= \frac{l^2}{2t_0}
\end{align*}

\textbf{Example :} \\
\begin{itemize}
    \item $D = 100\mu m ^2 /s$
    \item $l_{bacteria} = 1 \mu m$
    \item $l_{cell} = 20 \mu m$
    \item $l_{arm} = 1m$
\end{itemize}
\[ t_{bacteria} = \frac{l^2}{2D} = \frac{(1 \mu m)^2}{2(100\mu m ^2 /s)} = 5 ms\]
\[ t_{cell} = \frac{l^2}{2D} = \frac{(20 \mu m)^2}{2(100\mu m ^2 /s)}= 2s \]
\[ t_{arm} = \frac{l^2}{2D} = \frac{(1m)^2}{2(100\mu m ^2 /s)}= 300 years \]

\emph{Q :} How long does it take to tRNA to find a ribosome ?\\
\begin{itemize}
    \item $N_{tRNA} = 200 000$ molecules / bacteria
    \item the length of the bacteria is around $1\mu m$ long
    \item The volume of the bacteria : $V = 1 \mu m^3$
    \item $C_{tRNA} = \frac{200 000}{1}$
    \item $d=\sqrt[3]{\frac{1}{C_{tRNA}}} \approx 0.016\mu m$
    \item So $D= 100 \mu m /s^2 \rightarrow t=\frac{1^2}{2D} \approx 1\mu s$
    \item But ribosomes synthetize at 40 nucleotides per seconds, the the synthesis is about of 2 seconds for a 80 mucleotides tRNA.
\end{itemize}
So the ribosomes binds quickly to the tRNA, and takes longer to synthesize.\\

What is the concentration ? $C=\frac{Number~of~particles}{Volume} = 200 000 molec/\mu m$. Although, the concentration is not uniform is the bacteria.\\
What is flux ($j$) ? Flux the number of particules that pass through an area A per unit of time. $j \propto 2-1 = 1$
\[ \frac{\#~of~particles}{\text{time step} \times \text{area of the plane}} \]
\emph{Example :}
\begin{itemize}
    \item $N_{H_2 O} = 10^{10} molecules$
    \item $T = 3000 sec$
    \item $A = 6 \mu m^2$
\end{itemize}
\[ j =  \frac{10 ^ {10} }{3\times 10^3 \times 6} = 10^6 \frac{molecules}{s.\mu m^2} \]

Fick's Law
\[j = -D\frac{\partial c}{\partial x} \]
~\\
Darcy's Law
\[ v= -d\frac{\partial P}{\partial x} \text{d : mobility coeef} \]
~\\
Ohm's Law
\[ \overrightarrow{j_E} = \sigma \overrightarrow{E} = -\sigma \frac{\partial V}{\partial x} \text{ ~~~sigma : electric conductivity, V : electric potential} \]
~\\
Fourier's Law
\[ j_T = -K\frac{\partial T}{\partial x} \text{~~~T : temperature, K : thermal conductivity}\]
~\\

Random walk :
\[ j = \frac{particles(\rightarrow)-particles(\leftarrow)}{t_0 A} \]
\[ particles(\rightarrow) = N(x)\times \frac{1}{2} = C(x)V_{Box}\frac{1}{2} = C(x)Al\frac{1}{2} \]
\[ particles (\leftarrow) = N(x+l)\frac{1}{2} = C(x+l)Al\frac{1}{2} \]


\begin{align*}
    j &= \frac{C(x)\frac{l}{2} - C(x+l)\frac{l}{2}}{t_0}\\
    &= \frac{l^2}{2t_0}\left( \frac{C(x) - C(x+l)}{l} \right)\\
    &= -D\frac{\partial C}{\partial x}
\end{align*}

~\\
Law of mass conservation : $\#part@_{t+t_0} = \#part@_t + \#part_{entering} - \#part_{leaving}$\\
$\#part_{entering} = j(x)At_0$, $part_{leaving} = j(x+l)At_0$\\
$\#part@_{t+t_0} = C(x,t+t_0)lA$

\begin{align*}
(...) &= \left( \frac{C(x,t+t_0) - C(x,t)}{t_0} \right) lAt_0\\
&= \frac{\partial C}{\partial t}lAT_0\\
\frac{\partial C}{\partial t} &= -\frac{\partial j}{\partial x}
\end{align*}

\[ \frac{\partial C}{\partial t} = D\frac{\partial^2C}{\partial x^2} \]


\begin{align*}
\frac{\partial C}{\partial t} = 0,\frac{\partial^2 C}{\partial x^2}=0, \frac{\partial C}{\partial t} =0\\
\text{If there is a concentration gradient :}\\
C &= -Kx \text{with K the slope of the conentration gradient}\\
\text{This is a non-equilibrium state}\\
\frac{\partial C}{\partial x} &= -K\\
\frac{\partial C}{\partial t} &= 0\\
\frac{\partial ^2 C}{\partial x^2} &= 0\\
\end{align*}

\emph{Example 3 :} There is a film of water, and a drop of ink on it. We just take a plan to consider the diffusion. $d$ is the size of the drop.
\begin{align*}
    C(x,t) &= \frac{N}{\sqrt{4 \pi Dt}} e^{\frac{x^2}{4Dt}}
\end{align*}

\emph{Example 4 :} There are centain molecules inside a cell. There is much more molecules outside the cell. There is a flux a the interface of the membrane.
\begin{align*}
    j_M &= -P_m \Delta C \text{ with : } \Delta C = C_{ext} - C_{int}\\
    R &= 10 \mu m ; P_m = 20 \mu m/s\\
    \frac{\partial C_i}{\partial t} &= \frac{j_m A}{V} = \frac{P_mA}{V} (C_e -C_i)\\
    \frac{\partial C_i}{\partial t} &= -\frac{P_m A}{V}C_i + \frac{P_m A}{V}C_e \\
\end{align*}
\begin{align*}
    \tau &= \frac{V}{P_m A} = \frac{L^3}{\frac{L}{T}{L^2}} = T\\
    &= \frac{\frac{4}{3}\pi R^3}{P_m 4 \pi R^2} = \frac{R}{3P_m} = \frac{1}{6}sec
\end{align*}
\begin{align*}
    \frac{\partial C_i}{\partial t} &= -\frac{1}{t}(C_i - C_e) \\
    \rightarrow C_i(t) &= C_e - C_e e^{-t/\tau}
\end{align*}

\section{Bias Random walk}

This is the case where the probabiblity of going left is not the same than to go right. $K_+ \not= K_-$
\[ F = \gamma v \rightarrow v=\frac{F}{\gamma} \text{ avec } \gamma \text{ le coefficient de friction}\]

\begin{align*}
    j_{New} &= vC\\
    j &= -D \frac{\partial C}{\partial c} + vc\\
\end{align*}

\begin{align*}
    \Delta x = v\Delta t \text{ and } N = C\Delta x A = cA\Delta t v\\
    j &= \frac{\#_{part}}{time.area} = \frac{CvA\Delta t}{\Delta t A}\\
    &= Cv = c\frac{F}{\gamma}\\
\end{align*}
\begin{align*}
    j_{tot} &= -D \frac{\partial C}{\partial x} + Cv\\
    \text{Equilibrium situation :}\\
    0 &= -D\frac{\partial C}{\partial x} + C\frac{F}{\gamma}\\
    \text{With the force } F&=cst\\
    \frac{\partial C}{\partial x} &= \frac{CF}{\gamma} ; \frac{\partial C}{\partial x} = \frac{CF}{\gamma D}\\
    \frac{\partial C}{\partial x} &= \frac{C}{\lambda} \text{ as } \lambda = \frac{\gamma D}{F} \\
    \frac{\partial C}{\partial x} = \frac{C}{\lambda} ; \frac{\partial C}{C} = \frac{\partial x}{\lambda}\\
    ~~~\\
    \int \frac{\partial C}{C} &= \ln C - \ln C_0 = \ln \frac{C}{C_0}\\
    \int \frac{\partial x}{\lambda} &= \frac{x}{\lambda} \\
    ~~~\\
    \text{So : } \ln\frac{C}{C_0} = \frac{x}{\lambda}\\
    C &= C_0 e^{x/\lambda}
\end{align*}

Same case as above, but with $F = -Kx$\\
\begin{align*}
    j_{Tot} = 0 = -D\frac{\partial C}{\partial x} + C\frac{F}{\gamma} &= -D\frac{\partial C}{\partial x} - \frac{CKx}{\gamma} = C\\
    \text{Solution :}\\
    D\frac{\partial C}{\partial x} &= -\frac{KCx}{\gamma}; \frac{\partial C}{\partial x} = -\frac{KCx}{\gamma D}\\
    \frac{dC}{dx} &= \frac{-KCx}{\gamma D} ; \frac{dC}{C} = -\frac{KCx}{\gamma D}dx;\\
    ~~~~~\\
    \int \frac{dC}{C} &= \ln C - \ln C_0\\
    \int \frac{-Kx}{\gamma D}dx &= -\frac{K}{\gamma D}\frac{x^2}{2}\\
    ~~~~~\\
    \ln \frac{C}{C_0} &= -\frac{Kx^2}{2\gamma D} \Rightarrow \\
    C &= C_0 e^{\frac{-Kx}{2\gamma D}}
\end{align*}

We expect that : $ \sigma ^2 = \frac{\gamma D}{K}$


\begin{center}
    In a drop :
\end{center}

At a surface, the particles evaporate at a rate $\frac{1}{\tau}$\\

\begin{align*}
    N_{surface} &= C_0 A \lambda \\
    j &= \frac{N_{surface}}{A\tau} = \frac{C_0 \lambda}{\tau}\\
\end{align*}

\begin{align*}
    N(t + \Delta t) &= N(t) - jA\Delta t\\
    \Rightarrow \frac{dN}{Dt} &= -jA\\
    \text{And : } C_0 = \frac{N}{V} = cst ~ &;~ N=C_0V\\
    \rightarrow \frac{dN}{dt}&= C_0\frac{dV}{dt}\\
    \frac{dV}{dt} &= \frac{-jA}{C_0}\\
    &= \frac{C_0\lambda A}{C_0\tau} \\
    &= -\frac{\lambda A}{\tau}
\end{align*}

\begin{align*}
    \frac{dV}{dt} &= 4\pi R^2\frac{dR}{dt} = A\frac{dR}{dt}\\
\end{align*}

\begin{align*}
    \frac{dR}{dt}&=\frac{-\lambda}{\tau}\\
    R(t)&=R_0 \times - \frac{\lambda}{\tau}t
\end{align*}


\end{document}
