\documentclass[10pt,a4paper]{article}
\usepackage[utf8]{inputenc}
\usepackage[english]{babel}
\usepackage{amsmath}
\usepackage{amsfonts}
\usepackage{amssymb}
\usepackage{graphicx}
\usepackage[left=2cm,right=2cm,top=2cm,bottom=2cm]{geometry}



\author{Baptiste Rouger}
\title{Bootcamp : Physics}



\begin{document}


\maketitle

\begin{flushright}
carles.blanch-mercader@curie.fr
\end{flushright}

\tableofcontents

\newpage

\part{Brownian motion}

Diffusion give rise to \emph{material transport}. It gives a quantitative understanding of many phenomenon (permeability...).

\section{Historical aspects :}
\begin{description}
    \item [1834 :] Clapeyron proposed a phenomenological equation for the ideal gas : $PV = n R T$, \\$PV = N.K_B .T$ (N : number of particules, $K_B$ Boltzman constant $1.38\times 10^{-23} \frac{kg . m}{K . s^2})$
    \item [1828 :] Brown : studied pollen that was moving
    \item [1905 :] Einstein : $\langle x^2\rangle = 2DT$ (D : diffusion coefficient)
    \item [1908 :] Perrin : confirmed the Einstein guess
\end{description}

\section{Random-walk}

Simplest model that describes Brownian motion. A particle has the same probabiblity to move to the right and to the left. After two jumps, there is one way to 2 ($\frac{1}{4}$) and -2 ($\frac{1}{4}$), and 2 ways to 0 ($\frac{1}{2}$).\\
\[ \langle a \rangle = \frac{1}{m} \sum ^m _{i=1} a_i =\sum ^m _{i=1} p_ia_i \]
\[ \langle a^2 \rangle = \frac{1}{m} \sum ^m _{i=1} a_i^2 \]
$\langle a \rangle$ the average position, with m the number of realisations, i the i-th realisation and $a_i$ the i-th realisation.\\
In our example : $\langle a \rangle = \frac{1}{4} (2 + {-2} + 0 + 0) = 0$ and $\langle a^2 \rangle = \frac{1}{4} (2^2 + {-2}^2 + 0^2 + 0^2) = 2$\\

In the 3 jumps case :
\begin{itemize}
    \item 1 way to 3
    \item 1 way to -3
    \item 3 ways to 1
    \item 3 ways to -1
\end{itemize}
$\langle a \rangle = \frac{1}{8}( 3 + {-3} + 3 \times 1 + 3\times {-1} ) = 0$ and $\langle a^2 \rangle = \frac{1}{8}( 3^2 + {-3}^2 + (3 \times 1)^2 + (3\times {-1})^2 )=3$\\

\[ N-jumps = \langle a \rangle = 0\text{ and }\langle a^2 \rangle = N\]

If N is large enough, distribution of position is a binomial distribution.\\

If we say that the boxes have a $l$ length, then $x = a \times l$, $t_0$ is the time delay between successive jumps. So $\frac{t}
{t_0} = N$


\end{document}
