\documentclass{article}
\usepackage[english]{babel}
\usepackage{amsmath, amssymb, amsthm}

\title{Bootcamp : Maths}
\author{Baptiste Rouger}


\begin{document}

    \maketitle

    \tableofcontents

    \newpage

    \section{Session 1}


    \subsection{Calculus}

    Calculus : study of changes.\\

    2 types of calculus : differential and integral calculus.\\

    Differential : distance $\rightarrow$ speed\\
    Integration : speed $\rightarrow$ distance\\

    \subsubsection{Derivative}

    \paragraph{Definition}

    ~\\The derivative is the slope of the tangent line on a point.\\
    \textit{Definition :} The derivative of $f$ at $x_0$ is the slope of the line tangent to the graph of $f$ at $x_0$.\\
    \[ \lim_{\Delta x \to 0}; \frac{f(x_0 + \Delta x) - f(x_0)}{\Delta x} = f'(x_0) \]



    \paragraph{Some derivatives}
    $f(x) = \frac{1}{x} (x>0)  $\\
    $\frac{\Delta f}{\Delta x} =  \frac{\frac{1}{x_0 + \Delta x} - \frac{1}{x_0}}{\Delta x} = -\frac{1}{x_0^2 + x_0 \Delta x} $
    \[ f'(x_0) = \lim_{\Delta x \to 0}; -\frac{1}{x_0^2 + x_0 \Delta x} = -\frac{1}{x_0^2} \]
    ~\\~\\

    $f(x) = x^n$\\
    \[f'(x) = \lim_{\Delta x \to 0}; \frac{(x + \Delta x)^n - x^n}{\Delta x} = \lim_{\Delta x \to 0}; \frac{(\binom n0 x^n \Delta x ^0 + \binom n1 x^n \Delta x ^1 +... ) - x^n}{\Delta x} = nx^{n-1} \]

    ~\\
    $(f+g)'(x) = f'(x)+g'(x)$


    \paragraph{Continuity}
    ~\\
    Notations : $f'(x) = \frac{dy}{dx} = \frac{df}{dx} = Df = \frac{d}{dx}f$ = y'\\

    \noindent \textit{Example : } 400 ft , $y = 400 - 16t^2$\\
    $400 -16t^2 = 0$\\
    $t = 5$\\
    On average, -80 ft per second. But we want to know the instantaneous speed when the pumpkin hits the ground :\\
    $f'(x) = -32t$\\
    $f'(5) = -160 ft.s^{-1}$\\

    $f(x)$ is continuous at $x_0$ if
    \[ lim_{x \to x_0}; f(x) = f(x_0)  \Leftrightarrow \lim_{x \to x_0}; f(x) - f(x_0) = 0\]
    \textbf{Theorem :}
    If $f$ is differentiable at $x_0$ then $f$ is continuous at $x_0$
    \[\lim_{x \to x_0}; f(x) - f(x_0) = \lim_{x \to x_0};\frac{f(x) - f(x_0)}{x-x_0}(x-x_0) = f'(x_0).0 = 0 \]

    \paragraph{Derivatives of composed functions}

    \begin{align*} (u+v)'(x) &= \lim_{\Delta x \to 0}; \frac{(u+v)(x+\Delta x) - (u +v)(x)}{\Delta x}\\
        &= \lim_{\Delta x \to 0}; \frac{u(x+\Delta x) + v(x+\Delta x) - u(x) - v(x)}{\Delta x}\\
        &= u'(x) + v'(x)
    \end{align*}

    \begin{align*}
        (uv)' &= u'v + v'u\\
        (\frac{u}{v})' &= \frac{u'v - v'u}{v^2}
    \end{align*}


    \[ \lim_{x \to 0}; \frac{\sin x}{x} = 1\]
    \[    \lim_{x \to 0}; \frac{\cos x -1}{x} = 0\]

    \begin{align*}
        f'(x) &= \frac{d}{dx} \sin x\\
        &= \lim_{x+\Delta x} \frac{\sin(x+\Delta x) - \sin x}{\Delta x}\\
        &= \lim_{x+\Delta x} \frac{\sin x \cos \Delta x + \sin \Delta x \cos x - \sin x}{\Delta x}\\
        &= \lim_{x+\Delta x} \frac{\sin(\cos \Delta x -1)}{\Delta x} + \lim_{x+\Delta x} \frac{\cos x \sin \Delta x}{\Delta x}\\
        &= 0 + \cos x
    \end{align*}

    \paragraph{Chain rules}
    $ y= f(g(t))$\\
    \[ \frac{dy}{dt} = \frac{dy}{dx}\frac{dx}{dx} \]

    $y = f(x) = \sin x$\\
    $x = g(t) = t^2$\\
    $y = sin(t^2)$\\
    \[ \frac{d}{dt}(\sin t^2) = (\frac{dy}{dx})(\frac{dx}{dt})=(\cos x)(2t) = \cos(t^2)2t \]
    $f(g(x))' = g'(x)f'(g(x))$

    \paragraph{Implicit differntiation}

    $y = f(x) $\\
    $x^2 + y^2 = 1 $\\


    \begin{align*}
        \frac{d}{dx} (x^a) &= ax^{a-1}\\
        a &= m/n ; y = x^{m/n} ; y^n = x^m\\
        ny^{n-1}.\frac{dy}{dx} &= mx^{m-1}\\
        \frac{dy}{dx} &= \frac{m}{n} \frac{x^{m-1}}{y^{n-1}}\\
        &= \frac{m}{n} \frac{x^{m-1}}{x^{{m-1}^{n-1}}}\\
        &= \frac{m}{n} \frac{x^{m-1}}{x^{\frac{m(n-1)}{n}}}\\
        \frac{m}{n}x^{\frac{m}{n} -1} &= ax^{a-1}
    \end{align*}

    \begin{center}
        $\cos a + b = \sin a \cos b + \cos a \sin b$
        $\sin a+b = \cos a \cos b + \sin a \sin b$
    \end{center}

    We can note derivative with $D^n$ with $n$ the degree of the derivative.\\
    $D^2 x^n = n(n-1)x^{n-2}$ so generalizing :
    \begin{align*}
        D^{n-1} x^n &= n(n-1)...2 x^{n-(n-1)}\\
        &= n! x
    \end{align*}

    For a bacterial populaiton, $y_0$ is the initial site of the population, $t$ is the time in days. The bactérias double every days.\\
    $y(t) = 2^t y_0$\\~\\

    $A_0$ is the inital amount, $n$ the number of years, the interest rate is $5\%$.\\
    $A(n) = A_0 (1.05)^n$\\

    \begin{align*}
        \frac{d}{dx} a^x &= \lim {\Delta x \to 0} ; \frac{a^{x+\Delta x} - a^x}{\Delta x}\\
        &= \lim {\Delta x \to 0} ; \frac{a^x (a^{\Delta x} -1)}{\Delta x}\\
        &= a^x \lim {\Delta x \to 0} ; \frac{a^{\Delta x} -1}{\Delta x}\\
        with \frac{d}{dx} a^x = a^x M(a)\\
        \frac{d}{dx}a^x_{x=0} &= a^0 M(a) = M(a)\\
        So :
        \frac{d}{dx} e^x &= e^x
    \end{align*}
    $f(x) = 2^x$ initial function for wich we want to comute the derivative.\\
    $f(kx) = 2^{kx} = (2^k)^x = b^x$ (with $b=2^k$)\\
    So $\frac{d}{dx}b^x = kf'(kx)$
    \[ \frac{d}{dx}b^x_{x=0} = kf'(0) = M(2)k \]
    If $y=e^x$ then $\ln y = x$\\

    \begin{align*}
        \frac{d}{dx} \ln x\\
        w = \ln x \Rightarrow e^w = x\\
        \frac{d}{dx} e^w &= \frac{d}{dx} x = 1\\
        \frac{d}{dx} e^w \frac{dw}{dx} &= 1\\
        e^w \frac{dw}{dx} &= 1 \Rightarrow \\
        \frac{dw}{dx} &= \frac{1}{e^w} = \frac{1}{x}
    \end{align*}

    \begin{align*}
        \frac{d}{dx} a^x &= ?\\
        a^x &= (e^{\ln a})^x = e^{x\ln a}\\
        \frac{d}{dx} a^x &= e^{x\ln a} . \ln a\\
        &= a^x \ln a\\
        M(a) &= \ln a
    \end{align*}

    \begin{align*}
        \frac{d}{dx} u &= ?\\
        \frac{d}{dx} \ln u &= \frac{d \ln u}{du} . \frac{du}{dx} \\
        &= \frac{1}{u}.\frac{du}{dx}\\
        \frac{d}{dx} a^x = ? , u = a^x , \ln u = x\ln a \\
        (\ln u)' &= \ln a. \frac{u'}{u}\\
        &= \ln a . u'\\
        &= \ln a .u \\
        (a^x)' &= a^x \ln a
    \end{align*}

    \subsection{Differential equations}

    \subsection{Antiderivatives}
    \label{sub:Antiderivatives}

    $\int G(x)= g(x)dx $ and $G'(x)=g(x)$\\

    \begin{align*}
        \int \sin x dx &= \cos x + C\\
        Find G(x) such as G'(x)= \sin x
    \end{align*}

    \begin{align*}
        \int x^a dx &= \frac{x^{a+1}}{a+1}+C\\
        (\frac{x^{a+1}}{a+1}+C)' &= \frac{1}{a+1}.(a+1).x^a + 0\\
        &= x^a\\
        ~\\
        If a = -1, x^a = x^{-1} = \frac{1}{x}\\
        \int \frac{dx}{x} &= \ln |x| + C\\
        (\ln(-x))' &= \frac{1}{-x}.(-1)\\
        &= \frac{1}{x}
    \end{align*}

    \textit{Theorem : } If $F' = G'$, then $F(x)=G(x)+C$ for all $x$.\\
    \textit{Proof :} If $F' = G'$, then $(F - G)' = F' - G' = 0$\\
    Hence, $F(x)-G(x)$ is a constant. Derivative is 0 $\rightarrow$ function is constant (follows from the mean value theorem).\\

    \[
    \label{Mean value theorem}
    \frac{f(b) - f(a)}{b-a} = f'(c)~ ~ ~ ~
    f ~differentiable ~for~ a~ < ~x ~< ~b ~ ~ ~ ~
    f ~continuous ~for~ a \leq x \leq b
    \]

    \begin{align*}
        \int x^3 (x^4 + 2)^5 dx &= \\
        with~ u = x^4 +2 ~and ~du = 4x^3 dx
        &= \int \frac{u^5}{4}du\\
        &= \frac{1}{4}.\frac{1}{6}u^6+C\\
        &= \frac{u^6}{24}+C\\
        &= \frac{(x^4 +2)^6}{24}+C
    \end{align*}

    \textit{Example}
    \begin{align*}
        \int \frac{dx}{x\ln x} &= \int \frac{du}{u}\\
        &= \ln |u| + C\\
        &= \ln |\ln x| +C\\
        with~ u=\ln x, ~ du=\frac{1}{x}dx
    \end{align*}


    %% Graph with exponential, logarithm.

    \subsubsection{Differential equations}

    $\frac{dy}{dx} = f(x)$\\
    $y = \int f(x)dx$\\

    \textit{Example :}\\
    \begin{align*}
        \frac{d}{dy}y + xy &= 0 \\
        \Leftrightarrow \frac{dy}{dx}&=-xy\\
        \Rightarrow \frac{dy}{y} &= -xdx\\
        \int \frac{dy}{y} &= -\int xdx \\
        \ln |y| &= - \frac{x^2}{2}+C\\
        |y| &= e^{-\frac{x^2}{2}+C} \\
        &= e^{-\frac{x^2}{2}}.A ~ with ~ A=e^C
    \end{align*}

    \textit{Generalisation :}
    \begin{align*}
        \frac{dy}{dx} &= f(x)g(y)\\
        \frac{dy}{g(y)} &= f(x)dx\\
        H(y) &= F(x) +C \\
        y &= H^{-1}(F(x)+C)
    \end{align*}

    \textit{Examples :}
    \begin{align*}
        \frac{dy}{dx} &= (2x+5)^4\\
        y &= \int (2x+5)^4 dx\\
        &= \frac{1}{10}(2x+5)^5 +C ~ \text{If we know that }y(0)=1\\
        1 &= \frac{1}{10}(2\times 0 + 5)^5 +C \\
        \Rightarrow C &= 1-\frac{5^5}{10}
    \end{align*}

    \begin{align*}
        \frac{dy}{dx}&=(y+1)^{-1}\\
        \frac{dy}{(y+1)^{-1}} &= dx\\
        \int (y+1)dy &= \int dx\\
         \frac{y^2}{2}+y &= x + C
    \end{align*}

    \begin{align*}
        \frac{dy}{dx} &= xy^2\\
        \int y^{-2}dy &= \int xdx\\
        -y^{-1} &= \frac{x^2}{2} +C\\
        y &= - \frac{1}{\frac{x^2}{2}+C}
    \end{align*}

    \begin{align*}
        \frac{dy}{dx} &= 4xy \text{With }y(1)=3, \text{find } y(3)\\
        \frac{dy}{y} &= 4x dx\\
        \ln |y| &= \frac{4}{2}x^2 +C\\
        \text{for }y(1)=3 ~~ 2+C &= ln 3 \\
        C &= \ln 3 -2
        \text{So we have :} \ln |y| &= 2 \times 3^2 + \ln 3 -2\\
        &= 16 + \ln 3\\
        y &= e^{16 + \ln 3}
    \end{align*}



    \subsection{Linear Algebra}

    \part{Matrices}

    \section{Vectors}
    $ \overrightarrow{v} = (v_1,v_2,v_3...)$. \\
    \[
        \overrightarrow{u} + \overrightarrow{v} =
        \left(\begin{array}{c}
            x_u + x_v \\
            y_u + y_v
        \end{array}
        \right)
    \]

    \[
        \overrightarrow{u} . \overrightarrow{v} = ||u||\times ||v|| \times \cos \theta = (x_ux_v + y_uy_v)
    \]

    \[
    \left(\begin{array}{cc}
        1 & 0 \\
        0 & 2
    \end{array} \right)
    \left(\begin{array}{c}
    4 \\
    3
    \end{array} \right)
    =
    \left(\begin{array}{c}
    4 \\
    6
    \end{array}\right)
    \]

    \section{Matrices}

    \[ matrix~A =
    \left(\begin{array}{ccc}
        a_{11} & a_{12} & ... \\
        a_{21} & a_{22} & ...\\
        a_{31} & ...
    \end{array} \right)
    \]

    Tu multiply 2 matrices A and B : we need : $m \times n$ and $n\times p$ and $A.B \not= B.A$\\

    \[
    \left(\begin{array}{cc}
        7 & 4\\
        A & 2
    \end{array} \right)
    \left(\begin{array}{ccc}
        4 & 2 & 1\\
        -1 & 0 & 8
    \end{array} \right)
    =
    \left(\begin{array}{ccc}
        7.4 + 4.-1 & 7.2+4.0 & 7.1+4.8\\
        1.4+2.-1 & 1.2+2.0 & 1.1+2.8
    \end{array} \right)
    =
    \left(\begin{array}{ccc}
        24 & 14 & 39\\
        2 & 2 & 17
    \end{array} \right)
    \]

    \[
    A= \left(\begin{array}{ccc}
        1 & 2 & 3\\
        4 & 5 & 6\\
        7 & 8 & 9
    \end{array} \right)
    ; B =
    \left(\begin{array}{cc}
        3 & 2\\
        -5 & 1\\
        8 & 0
    \end{array} \right)
    \]
    \[
        A.B =
        \left(\begin{array}{cc}
            20 & 4\\
            35 & 13\\
            53 & 22
        \end{array} \right)
    \]

    \section{Square matrices}
    The identity matrix is a matrix the is empty but for 1 on her diagonal.\\
    $A.I= A ; I.A = A$\\
    Determinant of A :
    \[
    A=
    \left(\begin{array}{cc}
        a & b \\
        c & d
    \end{array} \right)
    \]
    \[
    det(A)=
    \left|\begin{array}{cc}
        a & b \\
        c & d
    \end{array} \right|
    = a.d - b.c
    \]

    \[
    det(A)=
    \left|\begin{array}{ccc}
        a & b & c\\
        d & e &f\\
        g & h & i
    \end{array} \right|
    = a.
    \left|\begin{array}{cc}
        e & f \\
        h & i
    \end{array} \right|
    -b.
    \left|\begin{array}{cc}
        d & f \\
        g & i
    \end{array} \right|
    +c.
    \left|\begin{array}{cc}
        d & e \\
        g & h
    \end{array} \right|
    \text{ and then do it again...}
    \]


    \[
    \overrightarrow{u}=
    \left|\begin{array}{c}
        u_1\\
        u_2\\
        u_3
    \end{array} \right|,
    \overrightarrow{v}=
    \left|\begin{array}{c}
        v_1\\
        v_2\\
        v_3
    \end{array} \right|
    \]

    \[
        \overrightarrow{u} \otimes \overrightarrow{v}=
        \left|\begin{array}{ccc}
            \hat{x} & \hat{y} & \hat{z}\\
            u_1 & u_2 & u_3\\
            v_1 & v_2 & v_3
        \end{array} \right| =
            \hat{x}n - \hat{y}m + \hat{z}p
    \]
    With n,m and p the determinants of the matrix.

    \section{Eigenvalues and eigenvectors}

    \[ A\overrightarrow{v} = k\overrightarrow{v} ? \]
    A matrix, $\overrightarrow{v}$ a vector and $k$ a number.\\
    \begin{align*}
        A\overrightarrow{v} &= k\overrightarrow{v}\\
        A\overrightarrow{v} &= kI\overrightarrow{v}\\
        A\overrightarrow{v}-kI\overrightarrow{v}&=0\\
        (A-kI)\overrightarrow{v} &= 0\\
        \text{So : } det(A-kI) &= 0
    \end{align*}

    \begin{align*}
        A=
        \left(\begin{array}{cc}
            0 & 1\\
            -2 & 3
        \end{array} \right)&=
        \left|\begin{array}{cc}
            -k & 1\\
            -2 & 3-k
        \end{array} \right|\\
        &=
        -k\times(3-k) - 1 \times -2 = (k-2)(k-1) =0\\
        \text{So there is two eigenvalues : } k_1 &= 2 \text{ and } k_2 = 1
    \end{align*}

    \begin{align*}
        k &= 1\\
        A\overrightarrow{v} &= 1\overrightarrow{v}\\
        (A-1I)\overrightarrow{v}&=0\\
        \left(\begin{array}{cc}
            -1 & 1\\
            -2 & 2
        \end{array} \right)
        \left(\begin{array}{c}
            v_1\\
            v_2
        \end{array} \right) &= 0\\
        \left{ \begin{array}{c}
            -1v_1 + v_2 = 0\\
            -2v_1 + 2v_2 =0
        \end{array}\\
        v_2 = v1\\
        v_1\begin{array}{c}
            a\\
            a
        \end{array} &=
        a\begin{array}{c}
            1\\
            1
        \end{array}
    \end{align*}

\end{document}
