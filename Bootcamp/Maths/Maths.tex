\documentclass{article}
\usepackage[english]{babel}
\usepackage{amsmath, amssymb, amsthm}

\title{Bootcamp : Maths}
\author{Baptiste Rouger}


\begin{document}

    \maketitle

    \tableofcontents

    \newpage

    \section{Session 1}


        \subsection{Calculus}

            Calculus : study of changes.\\

            2 types of calculus : differential and integral calculus.\\

            Differential : distance $\rightarrow$ speed\\
            Integration : speed $\rightarrow$ distance\\

                \subsubsection{Derivative}

                    \paragraph{Definition}

                        ~\\The derivative is the slope of the tangent line on a point.\\
                        \textit{Definition :} The derivative of $f$ at $x_0$ is the slope of the line tangent to the graph of $f$ at $x_0$.\\
                        \[ \lim_{\Delta x \to 0}; \frac{f(x_0 + \Delta x) - f(x_0)}{\Delta x} = f'(x_0) \]



                    \paragraph{Some derivatives}
                    $f(x) = \frac{1}{x} (x>0)  $\\
                    $\frac{\Delta f}{\Delta x} =  \frac{\frac{1}{x_0 + \Delta x} - \frac{1}{x_0}}{\Delta x} = -\frac{1}{x_0^2 + x_0 \Delta x} $
                    \[ f'(x_0) = \lim_{\Delta x \to 0}; -\frac{1}{x_0^2 + x_0 \Delta x} = -\frac{1}{x_0^2} \]
                    ~\\~\\

                    $f(x) = x^n$\\
                    \[f'(x) = \lim_{\Delta x \to 0}; \frac{(x + \Delta x)^n - x^n}{\Delta x} = \lim_{\Delta x \to 0}; \frac{(\binom n0 x^n \Delta x ^0 + \binom n1 x^n \Delta x ^1 +... ) - x^n}{\Delta x} = nx^{n-1} \]

                    ~\\
                    $(f+g)'(x) = f'(x)+g'(x)$


                    \paragraph{Continuity}
                        ~\\
                        Notations : $f'(x) = \frac{dy}{dx} = \frac{df}{dx} = Df = \frac{d}{dx}f$ = y'\\

                        \noindent \textit{Example : } 400 ft , $y = 400 - 16t^2$\\
                        $400 -16t^2 = 0$\\
                        $t = 5$\\
                        On average, -80 ft per second. But we want to know the instantaneous speed when the pumpkin hits the ground :\\
                        $f'(x) = -32t$\\
                        $f'(5) = -160 ft.s^{-1}$\\

                        $f(x)$ is continuous at $x_0$ if
                        \[ lim_{x \to x_0}; f(x) = f(x_0)  \Leftrightarrow \lim_{x \to x_0}; f(x) - f(x_0) = 0\]
                        \textbf{Theorem :}
                            If $f$ is differentiable at $x_0$ then $f$ is continuous at $x_0$
                            \[\lim_{x \to x_0}; f(x) - f(x_0) = \lim_{x \to x_0};\frac{f(x) - f(x_0)}{x-x_0}(x-x_0) = f'(x_0).0 = 0 \]

                    \paragraph{Derivatives of composed functions}

                        \begin{align*} (u+v)'(x) &= \lim_{\Delta x \to 0}; \frac{(u+v)(x+\Delta x) - (u +v)(x)}{\Delta x}\\
                        &= \lim_{\Delta x \to 0}; \frac{u(x+\Delta x) + v(x+\Delta x) - u(x) - v(x)}{\Delta x}\\
                        &= u'(x) + v'(x)
                        \end{align*}

                        \begin{align*}
                            (uv)' &= u'v + v'u\\
                            (\frac{u}{v})' &= \frac{u'v - v'u}{v^2}
                        \end{align*}


                        \[ \lim_{x \to 0}; \frac{\sin x}{x} = 1\]
                        \[    \lim_{x \to 0}; \frac{\cos x -1}{x} = 0\]

                        \begin{align*}
                            f'(x) &= \frac{d}{dx} \sin x\\
                            &= \lim_{x+\Delta x} \frac{\sin(x+\Delta x) - \sin x}{\Delta x}\\
                            &= \lim_{x+\Delta x} \frac{\sin x \cos \Delta x + \sin \Delta x \cos x - \sin x}{\Delta x}\\
                            &= \lim_{x+\Delta x} \frac{\sin(\cos \Delta x -1)}{\Delta x} + \lim_{x+\Delta x} \frac{\cos x \sin \Delta x}{\Delta x}\\
                            &= 0 + \cos x
                        \end{align*}

                    \paragraph{Chain rules}
                        $ y= f(g(t))$\\
                        \[ \frac{dy}{dt} = \frac{dy}{dx}\frac{dx}{dx} \]

                        $y = f(x) = \sin x$\\
                        $x = g(t) = t^2$\\
                        $y = sin(t^2)$\\
                        \[ \frac{d}{dt}(\sin t^2) = (\frac{dy}{dx})(\frac{dx}{dt})=(\cos x)(2t) = \cos(t^2)2t \]
                        $f(g(x))' = g'(x)f'(g(x))$

                    \paragraph{Implicit differntiation}

                    $y = f(x) $\\
                    $x^2 + y^2 = 1 $\\


                    \begin{align*}
                        \frac{d}{dx} (x^a) &= ax^{a-1}\\
                        a &= m/n ; y = x^{m/n} ; y^n = x^m\\
                        ny^{n-1}.\frac{dy}{dx} &= mx^{m-1}\\
                        \frac{dy}{dx} &= \frac{m}{n} \frac{x^{m-1}}{y^{n-1}}\\
                         &= \frac{m}{n} \frac{x^{m-1}}{x^{{m-1}^{n-1}}}\\
                         &= \frac{m}{n} \frac{x^{m-1}}{x^{\frac{m(n-1)}{n}}}\\
                         \frac{m}{n}x^{\frac{m}{n} -1} &= ax^{a-1}
                    \end{align*}

                    \begin{center}
                        $\cos a + b = \sin a \cos b + \cos a \sin b$
                        $\sin a+b = \cos a \cos b + \sin a \sin b$
                    \end{center}

                    We can note derivative with $D^n$ with $n$ the degree of the derivative.\\
                    $D^2 x^n = n(n-1)x^{n-2}$ so generalizing :
                    \begin{align*}
                        D^{n-1} x^n &= n(n-1)...2 x^{n-(n-1)}\\
                        &= n! x
                    \end{align*}

                    For a bacterial populaiton, $y_0$ is the initial site of the population, $t$ is the time in days. The bactérias double every days.\\
                    $y(t) = 2^t y_0$\\~\\

                    $A_0$ is the inital amount, $n$ the number of years, the interest rate is $5\%$.\\
                    $A(n) = A_0 (1.05)^n$\\

                    \begin{align*}
                        \frac{d}{dx} a^x &= \lim {\Delta x \to 0} ; \frac{a^{x+\Delta x} - a^x}{\Delta x}\\
                        &= \lim {\Delta x \to 0} ; \frac{a^x (a^{\Delta x} -1)}{\Delta x}\\
                        &= a^x \lim {\Delta x \to 0} ; \frac{a^{\Delta x} -1}{\Delta x}\\
                        with \frac{d}{dx} a^x = a^x M(a)\\
                        \frac{d}{dx}a^x_{x=0} &= a^0 M(a) = M(a)\\
                        So :
                        \frac{d}{dx} e^x &= e^x
                    \end{align*}
                    $f(x) = 2^x$ initial function for wich we want to comute the derivative.\\
                    $f(kx) = 2^{kx} = (2^k)^x = b^x$ (with $b=2^k$)\\
                    So $\frac{d}{dx}b^x = kf'(kx)$
                    \[ \frac{d}{dx}b^x_{x=0} = kf'(0) = M(2)k \]
                    If $y=e^x$ then $\ln y = x$\\

                    \begin{align*}
                        \frac{d}{dx} \ln x\\
                        w = \ln x \Rightarrow e^w = x\\
                        \frac{d}{dx} e^w &= \frac{d}{dx} x = 1\\
                        \frac{d}{dx} e^w \frac{dw}{dx} &= 1\\
                        e^w \frac{dw}{dx} &= 1 \Rightarrow \\
                        \frac{dw}{dx} &= \frac{1}{e^w} = \frac{1}{x}
                    \end{align*}

                    \begin{align*}
                        \frac{d}{dx} a^x &= ?\\
                        a^x &= (e^{\ln a})^x = e^{x\ln a}\\
                        \frac{d}{dx} a^x &= e^{x\ln a} . \ln a\\
                        &= a^x \ln a\\
                        M(a) &= \ln a
                    \end{align*}

                    \begin{align*}
                        \frac{d}{dx} u &= ?\\
                        \frac{d}{dx} \ln u &= \frac{d \ln u}{du} . \frac{du}{dx} \\
                        &= \frac{1}{u}.\frac{du}{dx}\\
                        \frac{d}{dx} a^x = ? , u = a^x , \ln u = x\ln a \\
                        (\ln u)' &= \ln a. \frac{u'}{u}\\
                        &= \ln a . u'\\
                        &= \ln a .u \\
                        (a^x)' &= a^x \ln a
                    \end{align*}

        \subsection{Differential equations}

            \subsection{Antiderivatives}
            \label{sub:Antiderivatives}

            $G(x)= g(x)dx $ and $G'(x)=g(x)$\\

            \begin{align*}
                \sin x dx &= \cos x + C\\
                Find G(x) such as G'(x)= \sin x
            \end{align*}



            %% Graph with exponential, logarithm.

        \subsection{Linear Algebra}

\end{document}
